%%%%%%%%%%%%%%%%%%%%%%%%%%%%%%%%%%%%%%%%%%%%%%%%%%%%%%%%%%%%%%
% Executive_summary
%%%%%%%%%%%%%%%%%%%%%%%%%%%%%%%%%%%%%%%%%%%%%%%%%%%%%%%%%%%%%%

\chapter*{Executive summary}\label{ch:summary}

Forest  and  Landscape  Restoration  (FLR) is  a  process of regaining ecological functionality and enhancing human well-being in previously forested landscapes. Yet, its outcomes for biodiversity, climate change mitigation, water and socio-economic dimensions remains scarce. The present report provides a review of FLR impacts on different aspects of biodiversity (section \ref{ch:biodiv}), climate change mitigation (section \ref{ch:carbon}), water (section \ref{ch:water}), and socioeconomic dimensions (section \ref{ch:socio}), and introduce on-going methodologies for developing spatial explicit models for supporting decision making in two of Brazilian biomes with the highest demands for large-scale restoration (Atlantic Forest and Amazon). It also produce preliminary key guidelines for decision makers and restoration practitioners on FLR and its impacts (section \ref{ch:recom}).

FLR outcomes for biodiversity depend on processes related with the (re) colonization, supplementation and maintenance of wildlife species and popu-lations in restored systems and surrounding landscapes. In section 2, we quantify these processes in terms of biodiversity restoration, species connectivity and species extinction risk. Therefore, we conducted a: (i) literature review on the impacts of different restoration methods on biodiversity for all Brazilian biomes, (ii) quatitative comparisons for biodiversity between nega-tive reference (e.g. agriculture and pasturelands), simple and biodiverse agroforestry systems with original reference systems (e.g. less disturbed or old-growth forests) for the Atlantic Forest, (iii) spatial analysis to illustrate the importance of incorporating landscape connectivity in FLR planning for biodiversity recovery in the Atlantic Forest, and (iv) spatial analysis to mapping the probability of extinction for endemic Atlantic Forest species as a function of the marginal contribution of each hectare restored to reducing species’ extinction probability.

FLR can play an important role in mitigating climate change as Tropical forest restoration can sequester large amounts of carbon from the atmosphere into the above and below-ground biomass and into the soil. In the section 3, we conducted a: (i) systematic literature review to understand how restoration initiatives in the Atlantic Forest have accounted for soil indicators in their planning and management decisions, (ii) literature review  on the impacts of different restoration methods on carbon stock for all Brazilian biomes, and (iii) spatial analysis to map the potential carbon sequestration by FLR in Brazil.

FLR has emerged as a feasible solution for water regulation and purification as forests can perform eco-hydrological functions, as regulation of water flow and maintenance of water quality. In the section 4, we study the FLR impacts on water quality and quantity by: (i) conducting a spatial analysis to investigate soil loss and sediment exportation to water under different FLR scenarios in the Atlantic Forest, (ii) conducting an international workshop to better understand the relationships between FLR and water, (iii) proposing a new approach to prioritize areas for FLR based on water quality improvement and a methodology to access the impact of large-scale restoration scenarios on Amazon and Atlantic Forest precipitation patterns.

The success or failure of a FLR project depends not only ecological, but also on socioeconomic factors as forests and trees contribute in multiple ways through a variety of ecosystem services to alleviate poverty, reduce food insecurity and support sustainable livelihoods. In the section 5, we conducted a: (i) systematic review consolidating the existing literature regarding restoration, socioeconomic benefits and ecosystem services provision, and (ii) propose two strategies to include socioeconomic aspects into spatial restoration prioritization in the Amazon and Atlantic Forest biomes. 

Finally, this report also provides 11 key preliminary recommendations for guiding decision makers and restoration practitioners on FLR and its impacts on the topics discussed above. These recommendations are critical to allow the sustainability and replicability of restoration projects. This report aim to help unlock the flow of financial investments needed to implement the ambitious Brazilian restoration targets and commitments.