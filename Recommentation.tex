


%----------------------------------------------------------------------------------------
%	CHAPTER: Recommendation
%----------------------------------------------------------------------------------------

\chapter{Preliminary Recommendations} \label{ch:recom}


Forest and landscape restoration aims to conserve biodiversity, safeguard essential ecosystem services for human well-being, and achieve social and economic benefits. Here, we recommend a series of best practices to guide decision makers, scientists and practitioners to achieve these goals and scale-up restoration. We identified these practices based on our results and on literature review focused on lessons learned and drivers for success of restoration initiatives. The ideal framework considers a transdisciplinary, participatory and adaptive management approach. 

The FLR mindset includes biophysical and socioeconomic aspects, as described on Table \ref{table:recommend}, and demands: (i) a better picture of social and environmental perceptions, (ii) multistakeholders involvement, (iii) socio-economic benefits evaluation and promotion, (iv) technical assistance, (v) spatial planning, (vi) attention to ecosystem services provision (as biodiversity conservation, carbon sequestration and water security), (vii) monitoring at multiple scales, (viii) communication and knowledge transfer. These practices can facilitate, stimulate and optimize restoration actions, reducing restoration costs and associated conflicts while optimizing its benefits.

\newpage

%%% to make table continues....
% \begin{table}
% \caption{A table}
% . . .
% \end{table}
% . . .
% \begin{table}\ContinuedFloat
% \caption{A table (cont.)}
% . . .
% \end{table}

{\small 
\begin{table} 
\caption{Main recommendations to achieve forest landscape restoration in Brazil.}
\begin{tabular}{|m{5.2cm}|m{5.5cm}|m{4.5cm}|}
\hline
\multicolumn{2}{|c|}{\bfseries Recommendation} & \bfseries Examples from literature    \\ 
\hline
\bfseries Restore with a landscape mindset      &Move beyond tree planting and incorporate both biophysical and socioeconomic aspects in the planning and implementation of restoration at the landscape level   &Dudley et al. 2002, Guariguata and Brancalion 2014, Brancalion et al. 2013, Mansurian et al. 2017     \\ 
\cline{1-3}
\multirow{2}{*}{\bfseries \makecell[l]{Capture and evaluate social and \\ environmental perceptions}}    &Understand the historical, cultural and economic backgrounds of each landscape and incorporate them at the outset of projects    &Ball et al. 2014,  Guariguata and Brancalion 2014 \\
\cline{2-3}   
&Evaluate the environmental perception and awareness of local communities   &Muler 2014, Lemgruber 2017  \\ 
\cline{1-3}
\multirow{2}{*}{\bfseries \makecell[l]{Foster bottom-up and \\ horizontal negociation \\ with stakholders}}  &Encourage local participation and involvement in restoration initiatives  &Le et al. 2012, Meli et al. 2017, Vieira et al. 2009, Ecker, 2016, Evans and Guariguata 2016 \\
\cline{2-3}  
& Integrate the aims and needs of different stakeholders   &Ball et al. 2014,  Guariguata and Brancalion 2014, Le et al. 2012, Brancalion et al. 2013, Meli et al. 2017 \\
\cline{2-3} 
& Promote space for negotiated decision-making  &Guariguata and Brancalion 2014, McGrath et al. 2008 \\
\cline{1-3}
\multirow{2}{*}{\bfseries \makecell[l]{Consider and promote socio-\\ economic benefits}}   &Promote integrated production systems (agroforestry, silvipasture, aquaculture)  &Le et al. 2012, Mansourian et al. 2014; Guariguata and Brancalion 2014, Jenkins at al. 2004, Menz et al. 2012, Meli et al. 2017, Veira et al. 2009, Ball et al. 2014  \\
\cline{2-3}
&Revise legal frameworks to broaden the possibilities for exploitation of native plant species  &Ball et al. 2014; Guariguata and  Brancalion 2014, Aronson 2010 \\
\cline{2-3}
&Promote marketing prospection of bidiversity products (timber and non-timber) and benefits from the restoration market chain     &Le et al. 2012, Jenkins et al. 2004, Ball and Brancalion 2016 \\
\cline{2-3}
&Encourage the creation of jobs in the restoration chain  &Calmon et al. 2011, ITPA 2010, IPEA 2015 \\
\cline{2-3}
&Encourage payment for ecosystem services polices   &Alves-Pinto et al. 2018, Grima et al. 2015, Zanella et al. 2014 \\
\cline{1-3}
\multirow{2}{*}{\bfseries Boost technical assistance}   &Improve rural extension and training  &Le et al. 2012, Pinto et al. 2014, Evans and Guariguata 2016  \\
\cline{2-3}
&Diversify restoration techniques according to local features  &Martins 2018  \\
\cline{2-3}
&Consider functional diversity when selecting species for planting  &Brancalion and Holl 2016  \\
\cline{2-3}
&Recognize natural regeneration as a relevant method for upscaling restoration  &Strassburg et al. 2016, Crouzeilles et al. 2017, Latawiec et al. 2016  \\
\hline
\end{tabular}





\label{table:recommend}
\end{table}
}
\newpage


{\small 
\begin{table} 
\ContinuedFloat 
\caption{Main recommendations to achieve forest landscape restoration in Brazil (Continued).}
%\label{table:recommend2}

% Continuação da tablerecommend
\begin{tabular}{|m{5.2cm}|m{5.5cm}|m{4.5cm}|}
\hline
\multicolumn{2}{|c|}{\bfseries Recommendation} & \bfseries Examples from literature    \\ 
\hline
\bfseries Boost technical assistance   &Promote rural innovations  &Pannell et al. 2006, Knight et al. 2010 \\
\cline{1-3}
\bfseries Consider spatial planning  &Prioritize FLR in areas with highest socio-ecological benefits per unit of cost  & Metzger et al. 2017, IIS 2017, Strassburg et al. 2018  \\
\cline{2-3}
&Identify scenarios focused on multiple outcomes  & Metzger et al. 2017, IIS 2017, Strassburg et al. 2018  \\ 
\cline{2-3}
&Consider the trade-offs and synergies among different restoration outcomes and targets to promote win-win solutions & Metzger et al. 2017, IIS 2017, Strassburg et al. 2018  \\ 
\cline{2-3}
&Incorporate climate changes projections into restoration planning, recognizing FLR as an ecosystem-based adaptation and mitigation to climate change  &Perry et al. 2015, Kane et al. 2017, Scarano and Ceotto 2015, Kasecker et al. 2017  \\
\cline{1-3}
\multirow{2}{*}{\bfseries \makecell[l]{Maximize biodiversity \\ conservation}}  &Increase landscape conectivity through stepping stones and corridors  &Crouzeilles et al. 2015, Rudnick et al. 2012, Tambosi et al. 2014, Watson et al. 2017  \\
\cline{2-3}
&Promote biodiversity-friendly land use systems to enhance matrix permeability to species dispersal, such as biodiverse rather than simplified agroforesty systems  &Watson et al. 2017, Donald and Evans 2006, Tambosi et al. 2014, Santos et al. 2018  \\
\cline{1-3}
\multirow{2}{*}{\bfseries Maximize carbon sequestration}  &Balance natural regeneration and active restoration to increase biomass accumulation over time  &Chazdon and Guariguata 2016, Crouzeilles et al. 2017  \\
\cline{2-3}
&Introduce species with greater rooting depth   &Stanturf et al. 2017  \\
\cline{2-3}
&Implement soil conservation measures to reduce erosion  &Stanturf et al. 2017  \\
\cline{2-3}
&Promote soil amendment to foster organic matter accumulation in the soil  &Stanturf et al. 2017  \\
\cline{1-3}
\multirow{2}{*}{\bfseries \makecell[l]{Maximize water quality \\ and provision}}  &Restore riparian and steepest slopes forests to prevent sediments from reaching the water bodies  &Saad et al. 2018  \\
\cline{2-3}
&Restore high altitude forests to improve water infiltration and reduce erosion, sedimentation, and downstream flooding  &Viviroli and Weingartner 2004, Bruijnzeel et al. 2011, Ghazoul and Sheil 2010, Ramírez et al. 2017  \\
\cline{2-3}
&Restore in areas where raises in rainfall are expected  &Ellison et al. 2017, Layton and Ellison 2016, Makarieva et al. 2006 \\
\hline 
\end{tabular}

\end{table}
} 

\newpage

{\small
\begin{table} 
\ContinuedFloat
\caption{Main recommendations to achieve forest landscape restoration in Brazil (Continued).}

%Continuacao (3) da Tabela recommend

\begin{tabular}{|m{5.2cm}|m{5.5cm}|m{4.5cm}|}
\hline
\multicolumn{2}{|c|}{\bfseries Recommendation} & \bfseries Examples from literature    \\ 
\hline
\multirow{2}{*}{\bfseries \makecell[l]{Implement long-term \\ monitoring at multiple scales}}  &Set specific targets, measurable goals, and objectives at the outset of projects  &Wortley et al. 2013  \\
\cline{2-3}
&Improve the application of time-series remote sensing monitoring on restoration projects  &Evans and Guariguata 2016  \\
\cline{2-3}
&Assess restoration quality and socioeconomic dimensions through on-the-ground monitoring over time  &FAO, CIFOR, IFRI \& World Bank 2016  \\
\cline{2-3}
&Encourage participatory monitoring   &Evans and Guariguata 2016, Ball and Brancalion 2016, Pinto et al. 2014, Meli et al.2017, Brancalion et al. 2013, Mansourian et al. 2017, McGrath et al. 2008  \\
\cline{1-3}
\multirow{2}{*}{\bfseries \makecell[l]{Foster communication and \\ knwoledge transfer}}   &Raise awareness of restoration benefits and limitations  &Terry et al. 2012, Brancalion et al. 2014 \\
\cline{2-3}
&Share knowledge in accessible communication frameworks  &Brancalion et al. 2013, Meli et al. 2017, Pinto et al. 2014, Evans and Guariguata 2016, Menz et al. 2013  \\
\cline{2-3}
&Promote knowledge exchange among stakeholders of different sectors and from different landscapes  &Terry et al. 2012, Brancalion et al. 2014  \\
\cline{2-3}
&Integrate educational and ecotourism activities in FLR iniciatives  &Muler 2014, Lemgruber 2017 \\
\hline 
\end{tabular}
%\label{table:recommend3}
\end{table}
}


% \section{Diretrizes para formuladores de políticas estaduais/federais}\label{sec:politcs}

% Well-preserved forests should be conserved and expanded, yet adopting more environmentally friendly land management practices in agricultural landscapes is a good complementary strategy. Policies focused on environmentally-friendly land management practices and FLR should advocate for the use of biodiverse agroforestry systems. The Native Vegetation Protection Law (N° 12.651/2012) allows the use of agroforestry systems to recover environmental debts in private rural properties, but do not recommend the type of agroforestry to be implemented. We recommend that biodiverse agroforestry should be favored over simple agroforestry systems as they can both provide higher values of biodiversity recovery (this report) and be more profitable (Miccolis et al. 2016).

% \section{Diretrizes para profissionais de restauracao}\label{sec:prof}

% We highlights the importance of incorporating landowners’ decision on forest restoration into spatial prioritization, but considering the ecological processes that occurs at the landscape scale, such as landscape connectivity. 
