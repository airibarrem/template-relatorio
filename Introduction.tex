

%%%%%%%%%%%%%%%%%%%%%%%%%%%%%%%%%%%%%%%%%%%%%%%%%%%%%%%%%%%%%%%%%%%%%%%%%%%%%%%%%%%%%%%%%%%%%%%%
% Introduction
%%%%%%%%%%%%%%%%%%%%%%%%%%%%%%%%%%%%%%%%%%%%%%%%%%%%%%%%%%%%%%%%%%%%%%%%%%%%%%%%%%%%%%%%%%%%%%%

\chapter{Forest Landscape Restoration in Brazil} \label{ch:intro}

The  high levels of deforestation and forest degradation, combined with the serious threats from climate change, have stimulated the international community to set international and country-led efforts aiming to boost Forest and Landscape Restoration worldwide (Box \ref{Box1}). The Aichi Targets 14 and 15 of the United Nations Convention on Biological Diversity, for example, aim to restore at least 15\% of degraded ecosystems, the Bonn Challenge and the New York Declaration on Forests of the United Nations 
 Climate Summit seeks to restore 150 and 350 M ha of degraded and deforested lands by 2020 and 2030, respectively \citep{Chazdon2017d}. Other remarkable efforts are the Initiative 20x20 in Latin America and the AFR100  in Africa, which seeks to restore 20 and 100 M ha of deforested and  degraded lands by 2020 and 2030, respectively \citep{Chazdon2017b}.
 
%%%%%%%
%%%%% BOX 1- FLR %%%%%%%%%%%%%%%%%%%%%


\begin{wrapfigure}[9]{r}{7.4cm}
\begin{mybox}{Defining  Forest and Landscape Restoration (FLR)}
\label{Box1}
FLR is a process of regaining ecological functionality and enhancing human well-being in previously forested landscapes (IUCN 2018).
\end{mybox}
\end{wrapfigure}
%
In the Brazilian context, two main instruments foster and regulate FLR: the Native Vegetation Protection Law (NVPL; Federal Law No 12,651/2012) and the National Plan for Native Vegetation Recovery (PLANAVEG). The NVPL is the main environmental law that protects the use of native vegetation in rural landholdings, replacing the previous Forest Code \citep{Soares-filho2014}. After a long period of discussion and debates, the new law established for the first time a governance structure and enforcement mechanisms for promoting the restoration of native ecosystems and the conservation of native ecosystems in rural landscapes. \\
\indent In a context of integrated landscape management, the NVPL determine that rural properties must conserve native forest, or restore them when necessary, in a portion of their land which depends on the Biome and property size (named Legal Reserve; for example, 20\% and 80\% in the Brazilian Atlantic Forest and Amazon biomes, respectively), and in areas with special ecologi-cal interest such as riparian forests, water springs, hilltops and slopes over 45$^{o}$ degrees (named Permanent Preserved Areas). Under the NVPL, agricultural credits will be restricted only to farmers that comply with the environmental law, as a way to enforce and guarantee restoration and conservation actions. The PLANAVEG is a top-down process conducted by the Brazilian Environmental Ministry, which joined effort with academia, private sector, NGOs, and state governments since 2013, to motivate and create the enabling conditions and incentives for rural landowners to restore at least 12.5 M ha of degraded and deforested lands by 2030 in Brazil \citep{Brasil2017}. This target is aligned with the estimated environmental debits enforced by the NVPL and the Brazil’s national and international commitments to the Aichi, Nationally Determined Contribution, the Bonn Challenge and the Initiative 20x20 targets. Therefore, Brazilian restoration practitioners, researchers, stakeholders and decision-makers face a key implementation challenge to reach the ambitious restoration targets set for the next decades.\\
%
\indent The Amazon and Atlantic Forest are the two Brazilian biomes with the highest demand for restoration of native vegetation, with each biome holding approximately 5 M ha of deforested and degraded land to be restored by law \citep{Brasil2017}. The Amazon and Atlantic Forest biomes lost, respectively, almost 18\% and 68\% of their original vegetation \citep{MAPBIOMAS2018Collection2000-2016}. Deforestation has been mainly driven by the expansion of agriculture and pasturelands, having important impacts on species conservation and on the emission of greenhouse gases to the atmosphere. The region embraced by the Atlantic Forest biome was responsible in 2017 for 24\% of the Brazilian greenhouse gas emissions, which were mainly produced by the agriculture and energy sectors \citep{SEEG2018TabelaDados}. The Amazonian region contributes to 33\% of Brazilian emissions and those are mainly produced by agricultural activities and land-use change \citep{SEEG2018TabelaDados}. Land use change also leads to the reduction of habitat for native species threatening biodiversity conservation. It is currently estimated that in the Amazon 183 fauna species are endangered (being 122 endemics), while in the Atlantic Forest this number reaches 589 fauna species (428 endemic). These biomes also play important roles in the national economy, given that it shelters 63\% of the Brazilian population (ca. 10\% and 53\% in the Amazon and Atlantic Forest) and almost 61.4\% of the Brazilian GDP (7.1\% and 54.3\% in the Amazon and Atlantic Forest, respectively) \citep{20182018Contas2018}. 

Although scaling up FLR is difficult, lengthy, expensive and budget-limited \citep{Crouzeilles2016}, it can offset some of the profound negative impacts of human development on ecosystems through the delivery of multiple bene-fits such as habitats for biodiversity, climate change mitigation, clean water provision, and sustainable livelihoods for people \citep{Chazdon2016c, Holl2017}. Consensus exists that each dollar invested in restoration needs to be spent in the most ecologically and economically efficient way \cite{Ding2017, Verdone2017}. 

The present report provides a review of FLR impacts on different aspects of biodiversity (section \ref{ch:biodiv}), climate change mitigation (section \ref{ch:carbon}), water (section \ref{ch:water}), and socioeconomic dimensions (section \ref{ch:socio}), and and introduce on-going methodologies for developing spatial explicit models for supporting decision making in two of Brazilian biomes with the highest demands for large-scale restoration (Atlantic Forest and Amazon). It also produce preliminary key guidelines for decision makers and restoration practitioners on FLR and its impacts (section \ref{ch:recom}). This report do not aim to extensively review all aspects from FLR impacts, but to better understand and spatially map key FLR impacts, which may help unlock the flow of financial investments needed to implement the ambitious Brazilian restoration targets and commitments.
