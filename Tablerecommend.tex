\begin{tabular}{|m{5.2cm}|m{5.5cm}|m{4.5cm}|}
\hline
\multicolumn{2}{|c|}{\bfseries Recommendation} & \bfseries Examples from literature    \\ 
\hline
\bfseries Restore with a landscape mindset      &Move beyond tree planting and incorporate both biophysical and socioeconomic aspects in the planning and implementation of restoration at the landscape level   &Dudley et al. 2002, Guariguata and Brancalion 2014, Brancalion et al. 2013, Mansurian et al. 2017     \\ 
\cline{1-3}
\multirow{2}{*}{\bfseries \makecell[l]{Capture and evaluate social and \\ environmental perceptions}}    &Understand the historical, cultural and economic backgrounds of each landscape and incorporate them at the outset of projects    &Ball et al. 2014,  Guariguata and Brancalion 2014 \\
\cline{2-3}   
&Evaluate the environmental perception and awareness of local communities   &Muler 2014, Lemgruber 2017  \\ 
\cline{1-3}
\multirow{2}{*}{\bfseries \makecell[l]{Foster bottom-up and \\ horizontal negociation \\ with stakholders}}  &Encourage local participation and involvement in restoration initiatives  &Le et al. 2012, Meli et al. 2017, Vieira et al. 2009, Ecker, 2016, Evans and Guariguata 2016 \\
\cline{2-3}  
& Integrate the aims and needs of different stakeholders   &Ball et al. 2014,  Guariguata and Brancalion 2014, Le et al. 2012, Brancalion et al. 2013, Meli et al. 2017 \\
\cline{2-3} 
& Promote space for negotiated decision-making  &Guariguata and Brancalion 2014, McGrath et al. 2008 \\
\cline{1-3}
\multirow{2}{*}{\bfseries \makecell[l]{Consider and promote socio-\\ economic benefits}}   &Promote integrated production systems (agroforestry, silvipasture, aquaculture)  &Le et al. 2012, Mansourian et al. 2014; Guariguata and Brancalion 2014, Jenkins at al. 2004, Menz et al. 2012, Meli et al. 2017, Veira et al. 2009, Ball et al. 2014  \\
\cline{2-3}
&Revise legal frameworks to broaden the possibilities for exploitation of native plant species  &Ball et al. 2014; Guariguata and  Brancalion 2014, Aronson 2010 \\
\cline{2-3}
&Promote marketing prospection of bidiversity products (timber and non-timber) and benefits from the restoration market chain     &Le et al. 2012, Jenkins et al. 2004, Ball and Brancalion 2016 \\
\cline{2-3}
&Encourage the creation of jobs in the restoration chain  &Calmon et al. 2011, ITPA 2010, IPEA 2015 \\
\cline{2-3}
&Encourage payment for ecosystem services polices   &Alves-Pinto et al. 2018, Grima et al. 2015, Zanella et al. 2014 \\
\cline{1-3}
\multirow{2}{*}{\bfseries Boost technical assistance}   &Improve rural extension and training  &Le et al. 2012, Pinto et al. 2014, Evans and Guariguata 2016  \\
\cline{2-3}
&Diversify restoration techniques according to local features  &Martins 2018  \\
\cline{2-3}
&Consider functional diversity when selecting species for planting  &Brancalion and Holl 2016  \\
\cline{2-3}
&Recognize natural regeneration as a relevant method for upscaling restoration  &Strassburg et al. 2016, Crouzeilles et al. 2017, Latawiec et al. 2016  \\
\hline
\end{tabular}




